%\iffalse
%<*package>
%% \CharacterTable
%%  {Upper-case    \A\B\C\D\E\F\G\H\I\J\K\L\M\N\O\P\Q\R\S\T\U\V\W\X\Y\Z
%%   Lower-case    \a\b\c\d\e\f\g\h\i\j\k\l\m\n\o\p\q\r\s\t\u\v\w\x\y\z
%%   Digits        \0\1\2\3\4\5\6\7\8\9
%%   Exclamation   \!     Double quote  \"     Hash (number) \#
%%   Dollar        \$     Percent       \%     Ampersand     \&
%%   Acute accent  \'     Left paren    \(     Right paren   \)
%%   Asterisk      \*     Plus          \+     Comma         \,
%%   Minus         \-     Point         \.     Solidus       \/
%%   Colon         \:     Semicolon     \;     Less than     \<
%%   Equals        \=     Greater than  \>     Question mark \?
%%   Commercial at \@     Left bracket  \[     Backslash     \\
%%   Right bracket \]     Circumflex    \^     Underscore    \_
%%   Grave accent  \`     Left brace    \{     Vertical bar  \|
%%   Right brace   \}     Tilde         \~}
%</package>
%\fi
% \iffalse
% Doc-Source file to use with LaTeX2e
% Copyright (C) 2015 Nicola Talbot, all rights reserved.
% \fi
% \iffalse
%<*driver>
\documentclass{ltxdoc}

\usepackage{alltt}
\usepackage{graphicx}
\usepackage[utf8]{inputenc}
\usepackage[T1]{fontenc}
\usepackage{datetime2-en-fulltext}
\usepackage[colorlinks,
            bookmarks,
            hyperindex=false,
            pdfauthor={Nicola L.C. Talbot},
            pdftitle={datetime2-en-fulltext.sty: English Full Text
Date and Time Styles}]{hyperref}


\CheckSum{649}

\renewcommand*{\usage}[1]{\hyperpage{#1}}
\renewcommand*{\main}[1]{\hyperpage{#1}}
\IndexPrologue{\section*{\indexname}\markboth{\indexname}{\indexname}}
\setcounter{IndexColumns}{2}

\newcommand*{\sty}[1]{\textsf{#1}}
\newcommand*{\opt}[1]{\texttt{#1}\index{#1=\texttt{#1}|main}}

\RecordChanges
\PageIndex
\CodelineNumbered

\begin{document}
\DocInput{datetime2-en-fulltext.dtx}
\end{document}
%</driver>
%\fi
%
%\MakeShortVerb{"}
%
%\title{datetime2-en-fulltext.sty: English Full Text Date and Time
%Styles}
%\author{Nicola L. C. Talbot}
%\date{2014-04-01 (v1.0)}
%\maketitle
%
%\begin{abstract}
%This is a supplementary package that provides English full text date and
%time styles for the datetime2.sty package. These styles are
%analogous to the styles provided by the old \sty{datetime}'s
%\cs{textdate} and \cs{oclock}. \textbf{These styles are not
%expandable styles.} This means that you can't use them in PDF
%bookmarks nor can you use them in case-changing commands such as
%\cs{MakeUppercase}.
%\end{abstract}
%
%\section{Introduction}
%
%This package loads the \sty{datetime2} package and the
%"english-base" language module, if not already loaded, and provides the
%date styles:
%\begin{description}
%\item[en-fulltext] Produces the date in the style:
%\DTMsetdatestyle{en-fulltext}\today. (Lower case ordinal and number
%words.) Commands such as \cs{Today} will start with a capital. For
%example: \Today.
%\item[en-FullText] Produces the date in the style:
%\DTMsetdatestyle{en-FullText}\today. (Title Case.)
%\item[en-FULLTEXT] Produces the date in the style:
%\DTMsetdatestyle{en-FULLTEXT}\today. (All caps.)
%\end{description}
%These styles honour the \opt{showdow} setting. Example:
%\begin{verbatim}
%\documentclass{article}
%\usepackage[showdow]{datetime2-en-fulltext}
%\begin{document}
%Date: \today.
%\end{document}
%\end{verbatim}
%The separator between the month and year is given by
%\cs{DTMenfulltextmonthyearsep} which may be redefined using
%\cs{renewcommand}.
%
%The \sty{datetime2-en-fulltext} package automatically switches on
%the "en-fulltext" date style. You can switch the style using
%\cs{DTMsetdatestyle}. For example:
%\begin{verbatim}
%\DTMsetdatestyle{en-FullText}
%\end{verbatim}
%
%This package also provides the time styles:
%\begin{description}
%\item[en-fulltext] Produces the time in the style:
%\DTMsettimestyle{en-fulltext}\DTMcurrenttime. (Lower case.)
%
%\item[en-Fulltext] Produces the time in the style:
%\DTMsettimestyle{en-Fulltext}\DTMcurrenttime. (Initial capital.)
%
%\item[en-FullText] Produces the time in the style:
%\DTMsettimestyle{en-FullText}\DTMcurrenttime. (Title Case.)
%
%\item[en-FULLTEXT] Produces the time in the style:
%\DTMsettimestyle{en-FULLTEXT}\DTMcurrenttime. (All caps.)
%
%\end{description}
%
%The \sty{datetime2-en-fulltext} package automatically switches on
%the "en-fulltext" time style. You can switch the style using
%\cs{DTMsettimestyle}. For example:
%\begin{verbatim}
%\DTMsettimestyle{en-FullText}
%\end{verbatim}
%
%Note that if you use \sty{polyglossia} or \sty{babel} and you don't
%have \sty{datetime2}'s \opt{useregional} setting set to "false",
%then the regional styles will override the styles provided here.
%
%\StopEventually{%
%\clearpage
%\phantomsection
%\addcontentsline{toc}{section}{Change History}%
%\PrintChanges
%\addcontentsline{toc}{section}{\indexname}%
%\PrintIndex}
%\section{The Code}
%
%\changes{1.0}{2015-04-01}{Initial release}
%\iffalse
%    \begin{macrocode}
%<*datetime2-en-fulltext.sty>
%    \end{macrocode}
%\fi
%Identify package
%    \begin{macrocode}
\NeedsTeXFormat{LaTeX2e}
\ProvidesPackage{datetime2-en-fulltext}[2015/04/01 v1.0]
%    \end{macrocode}
% Pass all options to \sty{datetime2}:
%    \begin{macrocode}
\DeclareOption*{\PassOptionsToPackage{\CurrentOption}{datetime2}}
\ProcessOptions
%    \end{macrocode}
% Requires \sty{datetime2}:
%    \begin{macrocode}
\RequirePackage{datetime2}
%    \end{macrocode}
% Also requries \sty{fmtcount}:
%    \begin{macrocode}
\RequirePackage{fmtcount}
%    \end{macrocode}
% If the "english-base" language module hasn't been loaded, load it now.
%    \begin{macrocode}
\def\CurrentTrackedDialect{english}
\RequireDateTimeModule{english-base}
\undef\CurrentTrackedDialect
%    \end{macrocode}
% Define the "en-fulltext" date style:
%    \begin{macrocode}
\DTMnewdatestyle{en-fulltext}{%
  \renewcommand*\DTMdisplaydate[4]{%
   \ifDTMshowdow
     \ifnum##4>-1
       \DTMenglishweekdayname{##4} the\space
     \fi
   \fi
   \protect\ordinalstringnum{##3} of \DTMenglishmonthname{##2}%
   \DTMenfulltextmonthyearsep
   \protect\numberstringnum{##1}%
  }%
  \renewcommand*\DTMDisplaydate[4]{%
   \ifDTMshowdow
     \ifnum##4>-1
       \DTMenglishweekdayname{##4} the\space
     \else
      \protect\Ordinalstringnum{##3} 
     \fi
   \else
     \protect\Ordinalstringnum{##3} 
   \fi
   of \DTMenglishmonthname{##2}%
   \DTMenfulltextmonthyearsep
   \protect\numberstringnum{##1}%
  }%
}
%    \end{macrocode}
%
% Define the "en-FullText" date style:
%    \begin{macrocode}
\DTMnewdatestyle{en-FullText}{%
  \renewcommand*\DTMdisplaydate[4]{%
   \ifDTMshowdow
     \ifnum##4>-1
       \DTMenglishweekdayname{##4} the\space
     \fi
   \fi
   \protect\Ordinalstringnum{##3} of \DTMenglishmonthname{##2}%
   \DTMenfulltextmonthyearsep
   \protect\Numberstringnum{##1}%
  }%
  \renewcommand*\DTMDisplaydate{\DTMdisplaydate}%
}
%    \end{macrocode}
%
% Define the "en-FULLTEXT" date style:
%    \begin{macrocode}
\DTMnewdatestyle{en-FULLTEXT}{%
  \renewcommand*\DTMdisplaydate[4]{%
   \ifDTMshowdow
     \ifnum##4>-1
       \MakeUppercase{\DTMenglishweekdayname{##4}} THE\space
     \fi
   \fi
   \protect\ORDINALstringnum{##3} OF 
   \MakeUppercase{\DTMenglishmonthname{##2}%
   \DTMenfulltextmonthyearsep}%
   \protect\NUMBERstringnum{##1}%
  }%
  \renewcommand*\DTMDisplaydate{\DTMdisplaydate}%
}
%    \end{macrocode}
%
%\begin{macro}{\DTMenfulltextmonthyearsep}
% Separator between month and year.
%    \begin{macrocode}
\newcommand*\DTMenfulltextmonthyearsep{,\space}
%    \end{macrocode}
%\end{macro}
% Set the "en-fulltext" date style:
%    \begin{macrocode}
\DTMsetdatestyle{en-fulltext}
%    \end{macrocode}
%
%Define the "en-fulltext" time style. Seconds are ignored.
%    \begin{macrocode}
\DTMnewtimestyle{en-fulltext}{%
  \renewcommand*\DTMdisplaytime[3]{%
   \ifboolexpr
   { test {\ifnumequal{##2}{0}} and 
    ( 
         test {\ifnumequal{##1}{0}}
      or test {\ifnumequal{##1}{12}}
      or test {\ifnumequal{##1}{24}}
    )
   }%
   {%
%    \end{macrocode}
% Either midnight or midday.
%    \begin{macrocode}
     \ifnum##1=12
       \DTMenglishnoon
     \else
       \DTMenglishmidnight
     \fi
   }%
   {%
%    \end{macrocode}
% Neither midnight nor midday.
%    \begin{macrocode}
     \ifnum##2=0
%    \end{macrocode}
% On the hour.
%    \begin{macrocode}
       \protect\numberstringnum{##1} \DTMoclockstring
     \else
%    \end{macrocode}
% Not on the hour.
%    \begin{macrocode}
       \ifnum##2<31
%    \end{macrocode}
% Past the hour.
%    \begin{macrocode}
         \ifnum##2=15
%    \end{macrocode}
% Quarter past the hour.
%    \begin{macrocode}
           \DTMquarterpaststring\space
         \else
           \ifnum##2=30
%    \end{macrocode}
% Half past the hour.
%    \begin{macrocode}
             \DTMhalfpaststring\space
           \else
             \protect\numberstringnum{##2}
             \ifnum##2=1 \DTMminutepaststring\else \DTMminutespaststring\fi
             \space
           \fi
         \fi
         \ifnum##1>12
           \ifnum##1=24
             \DTMenglishmidnight
           \else
             \protect\numberstringnum{\numexpr##1-12}%
           \fi
         \else
           \ifnum##1=0
             \DTMenglishmidnight
           \else
             \protect\numberstringnum{##1}%
           \fi
         \fi
       \else
%    \end{macrocode}
% To the hour.
%    \begin{macrocode}
         \ifnum##2=45
           \DTMquartertostring\space
         \else
           \protect\numberstringnum{\numexpr60-##2}
           \ifnum##2=59 \DTMminutetostring\else \DTMminutestostring\fi
           \space
         \fi
         \ifnum##1>12
           \ifnum##1=24
            \protect\numberstringnum{1}%
           \else
            \ifnum##1=23
              \DTMenglishmidnight
            \else
              \protect\numberstringnum{\numexpr##1-11}%
            \fi
           \fi
         \else
           \ifnum##1=12
            \protect\numberstringnum{1}%
           \else
            \protect\numberstringnum{\numexpr##1+1}%
           \fi
         \fi
       \fi
     \fi
     \ifnum##1>11
       \ifnum##1=23
         \ifnum##2<31
           \space\DTMafternoonstring
         \fi
       \else
         \ifnum##1=24
           \ifnum##2>30
             \space\DTMmorningstring
           \fi
         \else
           \space\DTMafternoonstring
         \fi
       \fi
     \else
       \ifnum##1>0
         \space\DTMmorningstring
       \else
         \ifnum##2>30
           \space\DTMmorningstring
         \fi
       \fi
     \fi
   }%
  }%
}
%    \end{macrocode}
%
%Define the "en-Fulltext" time style. Seconds are ignored.
%    \begin{macrocode}
\DTMnewtimestyle{en-Fulltext}{%
  \renewcommand*\DTMdisplaytime[3]{%
   \ifboolexpr
   { test {\ifnumequal{##2}{0}} and 
    ( 
         test {\ifnumequal{##1}{0}}
      or test {\ifnumequal{##1}{12}}
      or test {\ifnumequal{##1}{24}}
    )
   }%
   {%
%    \end{macrocode}
% Either midnight or midday.
%    \begin{macrocode}
     \ifnum##1=12
       \expandafter\MakeUppercase\DTMenglishnoon
     \else
       \expandafter\MakeUppercase\DTMenglishmidnight
     \fi
   }%
   {%
%    \end{macrocode}
% Neither midnight nor midday.
%    \begin{macrocode}
     \ifnum##2=0
%    \end{macrocode}
% On the hour.
%    \begin{macrocode}
       \protect\Numberstringnum{##1} \DTMoclockstring
     \else
%    \end{macrocode}
% Not on the hour.
%    \begin{macrocode}
       \ifnum##2<31
%    \end{macrocode}
% Past the hour.
%    \begin{macrocode}
         \ifnum##2=15
%    \end{macrocode}
% Quarter past the hour.
%    \begin{macrocode}
           \expandafter\MakeUppercase\DTMquarterpaststring\space
         \else
           \ifnum##2=30
%    \end{macrocode}
% Half past the hour.
%    \begin{macrocode}
             \expandafter\MakeUppercase\DTMhalfpaststring\space
           \else
             \protect\Numberstringnum{##2}
             \ifnum##2=1 \DTMminutepaststring\else \DTMminutespaststring\fi
             \space
           \fi
         \fi
         \ifnum##1>12
           \ifnum##1=24
             \DTMenglishmidnight
           \else
             \protect\numberstringnum{\numexpr##1-12}%
           \fi
         \else
           \ifnum##1=0
             \DTMenglishmidnight
           \else
             \protect\numberstringnum{##1}%
           \fi
         \fi
       \else
%    \end{macrocode}
% To the hour.
%    \begin{macrocode}
         \ifnum##2=45
           \expandafter\MakeUppercase\DTMquartertostring\space
         \else
           \protect\Numberstringnum{\numexpr60-##2}
           \ifnum##2=59 \DTMminutetostring\else \DTMminutestostring\fi
           \space
         \fi
         \ifnum##1>12
           \ifnum##1=24
            \protect\numberstringnum{1}%
           \else
            \ifnum##1=23
              \DTMenglishmidnight
            \else
              \protect\numberstringnum{\numexpr##1-11}%
            \fi
           \fi
         \else
           \ifnum##1=12
            \protect\numberstringnum{1}%
           \else
            \protect\numberstringnum{\numexpr##1+1}%
           \fi
         \fi
       \fi
     \fi
     \ifnum##1>11
       \ifnum##1=23
         \ifnum##2<31
           \space\DTMafternoonstring
         \fi
       \else
         \ifnum##1=24
           \ifnum##2>30
             \space\DTMmorningstring
           \fi
         \else
           \space\DTMafternoonstring
         \fi
       \fi
     \else
       \ifnum##1>0
         \space\DTMmorningstring
       \else
         \ifnum##2>30
           \space\DTMmorningstring
         \fi
       \fi
     \fi
   }%
  }%
}
%    \end{macrocode}
%
%Define the "en-FULLTEXT" time style. Seconds are ignored.
%    \begin{macrocode}
\DTMnewtimestyle{en-FULLTEXT}{%
  \renewcommand*\DTMdisplaytime[3]{%
   \ifboolexpr
   { test {\ifnumequal{##2}{0}} and 
    ( 
         test {\ifnumequal{##1}{0}}
      or test {\ifnumequal{##1}{12}}
      or test {\ifnumequal{##1}{24}}
    )
   }%
   {%
%    \end{macrocode}
% Either midnight or midday.
%    \begin{macrocode}
     \ifnum##1=12
       \MakeUppercase\DTMenglishnoon
     \else
       \MakeUppercase\DTMenglishmidnight
     \fi
   }%
   {%
%    \end{macrocode}
% Neither midnight nor midday.
%    \begin{macrocode}
     \ifnum##2=0
%    \end{macrocode}
% On the hour.
%    \begin{macrocode}
       \protect\NUMBERstringnum{##1} \MakeUppercase\DTMoclockstring
     \else
%    \end{macrocode}
% Not on the hour.
%    \begin{macrocode}
       \ifnum##2<31
%    \end{macrocode}
% Past the hour.
%    \begin{macrocode}
         \ifnum##2=15
%    \end{macrocode}
% Quarter past the hour.
%    \begin{macrocode}
           \MakeUppercase\DTMquarterpaststring\space
         \else
           \ifnum##2=30
%    \end{macrocode}
% Half past the hour.
%    \begin{macrocode}
             \MakeUppercase\DTMhalfpaststring\space
           \else
             \protect\NUMBERstringnum{##2}
             \MakeUppercase
             {%
               \ifnum##2=1 \DTMminutepaststring\else \DTMminutespaststring\fi
             }%
             \space
           \fi
         \fi
         \ifnum##1>12
           \ifnum##1=24
             \MakeUppercase\DTMenglishmidnight
           \else
             \protect\NUMBERstringnum{\numexpr##1-12}%
           \fi
         \else
           \ifnum##1=0
             \MakeUppercase\DTMenglishmidnight
           \else
             \protect\NUMBERstringnum{##1}%
           \fi
         \fi
       \else
%    \end{macrocode}
% To the hour.
%    \begin{macrocode}
         \ifnum##2=45
           \MakeUppercase\DTMquartertostring\space
         \else
           \protect\NUMBERstringnum{\numexpr60-##2}
           \MakeUppercase
           {%
             \ifnum##2=59 \DTMminutetostring\else \DTMminutestostring\fi
           }%
           \space
         \fi
         \ifnum##1>12
           \ifnum##1=24
            \protect\NUMBERstringnum{1}%
           \else
            \ifnum##1=23
              \MakeUppercase\DTMenglishmidnight
            \else
              \protect\NUMBERstringnum{\numexpr##1-11}%
            \fi
           \fi
         \else
           \ifnum##1=12
            \protect\NUMBERstringnum{1}%
           \else
            \protect\NUMBERstringnum{\numexpr##1+1}%
           \fi
         \fi
       \fi
     \fi
     \ifnum##1>11
       \ifnum##1=23
         \ifnum##2<31
           \space\MakeUppercase\DTMafternoonstring
         \fi
       \else
         \ifnum##1=24
           \ifnum##2>30
             \space\MakeUppercase\DTMmorningstring
           \fi
         \else
           \space\MakeUppercase\DTMafternoonstring
         \fi
       \fi
     \else
       \ifnum##1>0
         \space\MakeUppercase\DTMmorningstring
       \else
         \ifnum##2>30
           \space\MakeUppercase\DTMmorningstring
         \fi
       \fi
     \fi
   }%
  }%
}
%    \end{macrocode}
%
%Define the "en-FullText" time style. Seconds are ignored.
%    \begin{macrocode}
\DTMnewtimestyle{en-FullText}{%
  \renewcommand*\DTMdisplaytime[3]{%
   \ifboolexpr
   { test {\ifnumequal{##2}{0}} and 
    ( 
         test {\ifnumequal{##1}{0}}
      or test {\ifnumequal{##1}{12}}
      or test {\ifnumequal{##1}{24}}
    )
   }%
   {%
%    \end{macrocode}
% Either midnight or midday.
%    \begin{macrocode}
     \ifnum##1=12
       \expandafter\MakeUppercase\DTMenglishnoon
     \else
       \expandafter\MakeUppercase\DTMenglishmidnight
     \fi
   }%
   {%
%    \end{macrocode}
% Neither midnight nor midday.
%    \begin{macrocode}
     \ifnum##2=0
%    \end{macrocode}
% On the hour.
%    \begin{macrocode}
       \protect\Numberstringnum{##1} \DTMOClockstring
     \else
%    \end{macrocode}
% Not on the hour.
%    \begin{macrocode}
       \ifnum##2<31
%    \end{macrocode}
% Past the hour.
%    \begin{macrocode}
         \ifnum##2=15
%    \end{macrocode}
% Quarter past the hour.
%    \begin{macrocode}
           \DTMQuarterPaststring\space
         \else
           \ifnum##2=30
%    \end{macrocode}
% Half past the hour.
%    \begin{macrocode}
             \DTMHalfPaststring\space
           \else
             \protect\Numberstringnum{##2}
             \ifnum##2=1 \DTMMinutePaststring\else \DTMMinutesPaststring\fi
             \space
           \fi
         \fi
         \ifnum##1>12
           \ifnum##1=24
             \expandafter\MakeUppercase\DTMenglishmidnight
           \else
             \protect\Numberstringnum{\numexpr##1-12}%
           \fi
         \else
           \ifnum##1=0
             \expandafter\MakeUppercase\DTMenglishmidnight
           \else
             \protect\Numberstringnum{##1}%
           \fi
         \fi
       \else
%    \end{macrocode}
% To the hour.
%    \begin{macrocode}
         \ifnum##2=45
           \DTMQuarterTostring\space
         \else
           \protect\Numberstringnum{\numexpr60-##2}
           \ifnum##2=59 \DTMMinuteTostring\else \DTMMinutesTostring\fi
           \space
         \fi
         \ifnum##1>12
           \ifnum##1=24
            \protect\Numberstringnum{1}%
           \else
            \ifnum##1=23
              \expandafter\MakeUppercase\DTMenglishmidnight
            \else
              \protect\Numberstringnum{\numexpr##1-11}%
            \fi
           \fi
         \else
           \ifnum##1=12
            \protect\Numberstringnum{1}%
           \else
            \protect\Numberstringnum{\numexpr##1+1}%
           \fi
         \fi
       \fi
     \fi
     \ifnum##1>11
       \ifnum##1=23
         \ifnum##2<31
           \space\DTMAfterNoonstring
         \fi
       \else
         \ifnum##1=24
           \ifnum##2>30
             \space\DTMMorningstring
           \fi
         \else
           \space\DTMAfterNoonstring
         \fi
       \fi
     \else
       \ifnum##1>0
         \space\DTMMorningstring
       \else
         \ifnum##2>30
           \space\DTMMorningstring
         \fi
       \fi
     \fi
   }%
  }%
}
%    \end{macrocode}
%
%\begin{macro}{\DTMoclockstring}
%    \begin{macrocode}
\newcommand*{\DTMoclockstring}{o'clock}
%    \end{macrocode}
%\end{macro}
%
%\begin{macro}{\DTMOClockstring}
%    \begin{macrocode}
\newcommand*{\DTMOClockstring}{O'Clock}
%    \end{macrocode}
%\end{macro}
%
%\begin{macro}{\DTMquarterpaststring}
%    \begin{macrocode}
\newcommand*{\DTMquarterpaststring}{quarter past}
%    \end{macrocode}
%\end{macro}
%
%\begin{macro}{\DTMQuarterPaststring}
%    \begin{macrocode}
\newcommand*{\DTMQuarterPaststring}{Quarter Past}
%    \end{macrocode}
%\end{macro}
%
%\begin{macro}{\DTMminutepaststring}
%    \begin{macrocode}
\newcommand*{\DTMminutepaststring}{minute past}
%    \end{macrocode}
%\end{macro}
%
%\begin{macro}{\DTMminutespaststring}
%    \begin{macrocode}
\newcommand*{\DTMminutespaststring}{minutes past}
%    \end{macrocode}
%\end{macro}
%
%\begin{macro}{\DTMMinutePaststring}
%    \begin{macrocode}
\newcommand*{\DTMMinutePaststring}{Minute Past}
%    \end{macrocode}
%\end{macro}
%
%\begin{macro}{\DTMMinutesPaststring}
%    \begin{macrocode}
\newcommand*{\DTMMinutesPaststring}{Minutes Past}
%    \end{macrocode}
%\end{macro}
%
%\begin{macro}{\DTMhalfpaststring}
%    \begin{macrocode}
\newcommand*{\DTMhalfpaststring}{half past}
%    \end{macrocode}
%\end{macro}
%
%\begin{macro}{\DTMHalfPaststring}
%    \begin{macrocode}
\newcommand*{\DTMHalfPaststring}{Half Past}
%    \end{macrocode}
%\end{macro}
%
%\begin{macro}{\DTMquartertostring}
%    \begin{macrocode}
\newcommand*{\DTMquartertostring}{quarter to}
%    \end{macrocode}
%\end{macro}
%
%\begin{macro}{\DTMQuarterTostring}
%    \begin{macrocode}
\newcommand*{\DTMQuarterTostring}{Quarter to}
%    \end{macrocode}
%\end{macro}
%
%\begin{macro}{\DTMminutetostring}
%    \begin{macrocode}
\newcommand*{\DTMminutetostring}{minute to}
%    \end{macrocode}
%\end{macro}
%
%\begin{macro}{\DTMMinuteTostring}
%    \begin{macrocode}
\newcommand*{\DTMMinuteTostring}{Minute to}
%    \end{macrocode}
%\end{macro}
%
%\begin{macro}{\DTMminutestostring}
%    \begin{macrocode}
\newcommand*{\DTMminutestostring}{minutes to}
%    \end{macrocode}
%\end{macro}
%
%\begin{macro}{\DTMMinutesTostring}
%    \begin{macrocode}
\newcommand*{\DTMMinutesTostring}{Minutes to}
%    \end{macrocode}
%\end{macro}
%
%\begin{macro}{\DTMmorningstring}
%    \begin{macrocode}
\newcommand*{\DTMmorningstring}{in the morning}
%    \end{macrocode}
%\end{macro}
%
%\begin{macro}{\DTMMorningstring}
%    \begin{macrocode}
\newcommand*{\DTMMorningstring}{in the Morning}
%    \end{macrocode}
%\end{macro}
%
%\begin{macro}{\DTMafternoonstring}
%    \begin{macrocode}
\newcommand*{\DTMafternoonstring}{in the afternoon}
%    \end{macrocode}
%\end{macro}
%
%\begin{macro}{\DTMAfterNoonstring}
%    \begin{macrocode}
\newcommand*{\DTMAfterNoonstring}{in the Afternoon}
%    \end{macrocode}
%\end{macro}
%
%
%Set the "en-fulltext" time style:
%    \begin{macrocode}
\DTMsettimestyle{en-fulltext}
%    \end{macrocode}
%
%\iffalse
%    \begin{macrocode}
%</datetime2-en-fulltext.sty>
%    \end{macrocode}
%\fi
%\Finale
\endinput
