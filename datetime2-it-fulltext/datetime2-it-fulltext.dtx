%\iffalse
%<*package>
%% \CharacterTable
%%  {Upper-case    \A\B\C\D\E\F\G\H\I\J\K\L\M\N\O\P\Q\R\S\T\U\V\W\X\Y\Z
%%   Lower-case    \a\b\c\d\e\f\g\h\i\j\k\l\m\n\o\p\q\r\s\t\u\v\w\x\y\z
%%   Digits        \0\1\2\3\4\5\6\7\8\9
%%   Exclamation   \!     Double quote  \"     Hash (number) \#
%%   Dollar        \$     Percent       \%     Ampersand     \&
%%   Acute accent  \'     Left paren    \(     Right paren   \)
%%   Asterisk      \*     Plus          \+     Comma         \,
%%   Minus         \-     Point         \.     Solidus       \/
%%   Colon         \:     Semicolon     \;     Less than     \<
%%   Equals        \=     Greater than  \>     Question mark \?
%%   Commercial at \@     Left bracket  \[     Backslash     \\
%%   Right bracket \]     Circumflex    \^     Underscore    \_
%%   Grave accent  \`     Left brace    \{     Vertical bar  \|
%%   Right brace   \}     Tilde         \~}
%</package>
%\fi
% \iffalse
% Doc-Source file to use with LaTeX2e
% Copyright (C) 2015 Nicola Talbot, all rights reserved.
% \fi
% \iffalse
%<*driver>
\documentclass{ltxdoc}

\usepackage{alltt}
\usepackage{graphicx}
\usepackage[utf8]{inputenc}
\usepackage[T1]{fontenc}
\usepackage[colorlinks,
            bookmarks,
            hyperindex=false,
            pdfauthor={Nicola L.C. Talbot},
            pdftitle={datetime2.sty Italian Module}]{hyperref}


\CheckSum{465}

\renewcommand*{\usage}[1]{\hyperpage{#1}}
\renewcommand*{\main}[1]{\hyperpage{#1}}
\IndexPrologue{\section*{\indexname}\markboth{\indexname}{\indexname}}
\setcounter{IndexColumns}{2}

\newcommand*{\sty}[1]{\textsf{#1}}
\newcommand*{\opt}[1]{\texttt{#1}\index{#1=\texttt{#1}|main}}

\RecordChanges
\PageIndex
\CodelineNumbered

\begin{document}
\DocInput{datetime2-italian.dtx}
\end{document}
%</driver>
%\fi
%
%\MakeShortVerb{"}
%
%\title{Italian Module for datetime2 Package}
%\author{Nicola L. C. Talbot (inactive)}
%\date{2015-03-27 (v1.0)}
%\maketitle
%
% This module is currently maintained and may be subject to change.
% The Current Maintainer of this work is Claudio Duchi claduc at gmail dot com
%\begin{abstract}
%This is the Italian language module for the \sty{datetime2}
%package. If you want to use the settings in this module you must
%install it in addition to installing \sty{datetime2}. If you use
%\sty{babel} or \sty{polyglossia}, you will need this module to
%prevent them from redefining \cs{today}. The \sty{datetime2}
% \opt{useregional} setting must be set to "text" or "numeric"
% for the language styles to be set.
% Alternatively, you can set the style in the document using
% \cs{DTMsetstyle}, but this may be changed by \cs{date}\meta{language}
% depending on the value of the \opt{useregional} setting.
%\end{abstract}
%\opt{showdow} setting.
%
%\StopEventually{%
%\clearpage
%\phantomsection
%\addcontentsline{toc}{section}{Change History}%
%\PrintChanges
%\addcontentsline{toc}{section}{\indexname}%
%\PrintIndex}
%\section{The Code}
%\iffalse
%    \begin{macrocode}
%<*datetime2-italian-utf8.ldf>
%    \end{macrocode}
%\fi
%\subsection{UTF-8}
%This file contains the settings that use UTF-8 characters. This
%file is loaded if XeLaTeX or LuaLaTeX are used. Please make sure
%your text editor is set to UTF-8 if you want to view this code.
%\changes{1.0}{2015-03-27}{Initial release}
% Identify module
%    \begin{macrocode}
\ProvidesDateTimeModule{italian-utf8}[2015/04/10 v1.1]
%    \end{macrocode}
%\begin{macro}{\DTMitalianordinal}
%    \begin{macrocode}
\newcommand*{\DTMitalianordinal}[1]{%
  \number#1
}%
%    \end{macrocode}
%\end{macro}
%
%\begin{macro}{\DTMitalianmonthname}
% Italian month names.
%    \begin{macrocode}
\newcommand*{\DTMitalianmonthname}[1]{%
  \ifcase#1
  gennaio%
  \or
  febbraio%
  \or
  marzo%
  \or
  aprile%
  \or
  maggio%
  \or
  giugno%
  \or
  luglio%
  \or
  agosto%
  \or
  settembre%
  \or
  ottobre%
  \or
  novembre%
  \or
  dicembre%
  \fi
}
%    \end{macrocode}
%\end{macro}
%
%
%
%\begin{macro}{\DTMitalianweekdayname}
%These are provided here but not currently used in the date format.
%    \begin{macrocode}
\newcommand*{\DTMitalianweekdayname}[1]{%
  \ifcase#1
  lunedì%
  \or
  martedì%
  \or
  mercoledì%
  \or
  giovedì%
  \or
  venerdì%
  \or
  sabato%
  \or
  domenica%
  \fi
}
%    \end{macrocode}
%\end{macro}
%
%
%\begin{macro}{\DTMitalianshortmonthname}
%short mont.
%    \begin{macrocode}
\newcommand*{\DTMitalianshortmonthname}[1]{%
	\ifcase#1
	gen.%
	\or
	feb.%
	\or
	mar.%
	\or
	apr.%
	\or
	mag.%
	\or
	giun.%
	\or
	lug.%
	\or
	aug.%
	\or
	set.%
	\or
	ott.%
	\or
	nov.%
	\or
	dic.%
	\fi
}
%    \end{macrocode}
%\end{macro}
%\begin{macro}{\DTMitalianshortweekdayname}
%short week day.
%    \begin{macrocode}
\newcommand*{\DTMitalianshortweekdayname}[1]{%
	\ifcase#1
	lun.%
	\or
	mar.%
	\or
	mer.%
	\or
	gio.%
	\or
	ven.%
	\or
	sab.%
	\or
	dom.%
	\fi
}
%    \end{macrocode}
%\end{macro}
%\iffalse
%    \begin{macrocode}
%</datetime2-italian-utf8.ldf>
%    \end{macrocode}
%\fi
%\iffalse
%    \begin{macrocode}
%<*datetime2-italian-ascii.ldf>
%    \end{macrocode}
%\fi
%\subsection{ASCII}
%This file contains the settings that use \LaTeX\ commands for
%non-ASCII characters. This should be input if neither XeLaTeX nor
%LuaLaTeX are used. Even if the user has loaded \sty{inputenc} with
%"utf8", this file should still be used not the
%\texttt{datetime2-italian-utf8.ldf} file as the non-ASCII
%characters are made active in that situation and would need
%protecting against expansion.
%\changes{1.0}{2015-03-27}{Initial release}
% Identify module
%    \begin{macrocode}
\ProvidesDateTimeModule{italian-ascii}[2015/04/10 v1.1]
%    \end{macrocode}
%
%\begin{macro}{\DTMitalianordinal}
%    \begin{macrocode}
\newcommand*{\DTMitalianordinal}[1]{%
  \number#1
}%
%    \end{macrocode}
%\end{macro}
%
%\begin{macro}{\DTMitalianmonthname}
% Italian month names.
%    \begin{macrocode}
\newcommand*{\DTMitalianmonthname}[1]{%
  \ifcase#1
  \or
  gennaio%
  \or
  febbraio%
  \or
  marzo%
  \or
  aprile%
  \or
  maggio%
  \or
  giugno%
  \or
  luglio%
  \or
  agosto%
  \or
  settembre%
  \or
  ottobre%
  \or
  novembre%
  \or
  dicembre%
  \fi
}
%    \end{macrocode}
%\end{macro}
%
%
%\begin{macro}{\DTMitalianweekdayname}
%These are provided here but not currently used in the date format.
%    \begin{macrocode}
\newcommand*{\DTMitalianweekdayname}[1]{%
  \ifcase#1
  luned\protect\`i%
  \or
  marted\protect\`i%
  \or
  mercoled\protect\`i%
  \or
  gioved\protect\`i%
  \or
  venerd\protect\`i%
  \or
  sabato%
  \or
  domenica%
  \fi
}
%    \end{macrocode}
%\end{macro}
%
%
%\begin{macro}{\DTMitalianshortmonthname}
%short mont.
%    \begin{macrocode}
\newcommand*{\DTMitalianshortmonthname}[1]{%
	\ifcase#1
	\or
	gen.%
	\or
	feb.%
	\or
	mar.%
	\or
	apr.%
	\or
	mag.%
	\or
	giun.%
	\or
	lug.%
	\or
	aug.%
	\or
	set.%
	\or
	ott.%
	\or
	nov.%
	\or
	dic.%
	\fi
}
%    \end{macrocode}
%\end{macro}
%\begin{macro}{\DTMitalianshortweekdayname}
%short week day.
%    \begin{macrocode}
\newcommand*{\DTMitalianshortweekdayname}[1]{%
	\ifcase#1
	lun.%
	\or
	mar.%
	\or
	mer.%
	\or
	gio.%
	\or
	ven.%
	\or
	sab.%
	\or
	dom.%
	\fi
}
%    \end{macrocode}
%\end{macro}
%
%\iffalse
%    \begin{macrocode}
%</datetime2-italian-ascii.ldf>
%    \end{macrocode}
%\fi
%
%\subsection{Main Italian Module (\texttt{datetime2-italian.ldf})}
%\changes{1.0}{2015-03-10}{Initial release}
%\changes{1.1}{2015-04-10}{Add timestyle italianampm% }
%\changes{1.2}{2015-04-16}{Add timestyle italian-twenty-four% }
%
%\iffalse
%    \begin{macrocode}
%<*datetime2-italian.ldf>
%    \end{macrocode}
%\fi
%
% Identify Module
%    \begin{macrocode}
\ProvidesDateTimeModule{italian}[2015/04/16 v1.2]
%    \end{macrocode}
% Need to find out if XeTeX or LuaTeX are being used.
%    \begin{macrocode}
\RequirePackage{ifxetex,ifluatex}
%    \end{macrocode}
% XeTeX and LuaTeX natively support UTF-8, so load
% \texttt{italian-utf8} if either of those engines are used
% otherwise load \texttt{italian-ascii}.
%    \begin{macrocode}
\ifxetex
 \RequireDateTimeModule{italian-utf8}
\else
 \ifluatex
   \RequireDateTimeModule{italian-utf8}
 \else
   \RequireDateTimeModule{italian-ascii}
 \fi
\fi
%    \end{macrocode}
%
% Define the \texttt{italian} style.
% The time style is the same as the "default" style
% provided by \sty{datetime2}.
%
% the packet defines the styles: "italian",
% "italian-numeric", "italianampm" and "italian-twenty-four" styles. 
% 
%This doesn't use the package wide
% separators such as
% \cs{dtm@datetimesep} in case other date formats are also required.
%\begin{macro}{\DTMitaliandaymonthsep}
% The separator between the day and month for the text format.
%    \begin{macrocode}
\newcommand*{\DTMitaliandaymonthsep}{\space}
%    \end{macrocode}
%\end{macro}
%
%\begin{macro}{\DTMitalianmonthyearsep}
% The separator between the month and year for the text format.
%    \begin{macrocode}
\newcommand*{\DTMitalianmonthyearsep}{\space}
%    \end{macrocode}
%\end{macro}
%
%\begin{macro}{\DTMitaliandatetimesep}
% The separator between the date and time blocks in the full format
% (either text or numeric).
%    \begin{macrocode}
\newcommand*{\DTMitaliandatetimesep}{\space}
%    \end{macrocode}
%\end{macro}
%
%\begin{macro}{\DTMitaliantimezonesep}
% The separator between the time and zone blocks in the full format
% (either text or numeric).
%    \begin{macrocode}
\newcommand*{\DTMitaliantimezonesep}{\space}
%    \end{macrocode}
%\end{macro}
%
%\begin{macro}{\DTMitaliandatesep}
% The separator for the numeric date format.
%    \begin{macrocode}
\newcommand*{\DTMitaliandatesep}{/}
%    \end{macrocode}
%\end{macro}
%
%\begin{macro}{\DTMitaliantimesep}
% The separator for the numeric time format.
%    \begin{macrocode}
\newcommand*{\DTMitaliantimesep}{:}
%    \end{macrocode}
%\end{macro}
%
% Provide keys that can be used in \cs{DTMlangsetup} to set these
% separators.
%    \begin{macrocode}
\DTMdefkey{italian}{daymonthsep}{\renewcommand*{\DTMitaliandaymonthsep}{#1}}
\DTMdefkey{italian}{monthyearsep}{\renewcommand*{\DTMitalianmonthyearsep}{#1}}
\DTMdefkey{italian}{datetimesep}{\renewcommand*{\DTMitaliandatetimesep}{#1}}
\DTMdefkey{italian}{timezonesep}{\renewcommand*{\DTMitaliantimezonesep}{#1}}
\DTMdefkey{italian}{datesep}{\renewcommand*{\DTMitaliandatesep}{#1}}
\DTMdefkey{italian}{timesep}{\renewcommand*{\DTMitaliantimesep}{#1}}
%    \end{macrocode}

% Define a boolean key that determines if the shortened form
% should be used.
%    \begin{macrocode}
\DTMdefboolkey{italian}{abbr}[true]{}
%    \end{macrocode}
% The default is to use abbr
%    \begin{macrocode}
\DTMsetbool{italian}{abbr}{false}
%    \end{macrocode}
%
% Define a boolean key that determines if the time zone mappings
% should be used.
%    \begin{macrocode}
\DTMdefboolkey{italian}{mapzone}[true]{}
%    \end{macrocode}
% The default is to use mappings.
%    \begin{macrocode}
\DTMsetbool{italian}{mapzone}{true}
%    \end{macrocode}
%
% Define a boolean key that determines if the day of month should be
% displayed.
%    \begin{macrocode}
\DTMdefboolkey{italian}{showdayofmonth}[true]{}
%    \end{macrocode}
% The default is to show the day of month.
%    \begin{macrocode}
\DTMsetbool{italian}{showdayofmonth}{true}
%    \end{macrocode}
%
% Define a boolean key that determines if the year should be
% displayed.
%    \begin{macrocode}
\DTMdefboolkey{italian}{showyear}[true]{}
%    \end{macrocode}
% The default is to show the year.
%    \begin{macrocode}
\DTMsetbool{italian}{showyear}{true}
%    \end{macrocode}
%
%    \begin{macrocode}
\newcommand*\DTMitalianam{am}%
%    \end{macrocode}
%    \begin{macrocode}
\newcommand*\DTMitalianpm{pm}%
%    \end{macrocode}
%    \begin{macrocode}
\newcommand*\DTMitalianmidnight{mezzanotte}%
%    \end{macrocode}
%    \begin{macrocode}
\newcommand*\DTMitaliannoon{mezzogiorno}%
%    \end{macrocode}
%    \begin{macrocode}
\newcommand*{\DTMitalianampmfmt}[1]{#1}
%    \end{macrocode}
% Define the "italian" style. (TODO: implement day of week?)
%    \begin{macrocode}
\DTMnewstyle
{italian}% label
{% date style
%	\renewcommand*{\DTMitalianfmtordsuffix}{\DTMitalianfmtordsuffix}%
	\renewcommand*\DTMdisplaydate[4]{%
		\ifDTMshowdow
		\ifnum##4>-1%
		\DTMifbool{italian}{abbr}%
		{\DTMitalianshortweekdayname{##4}}%
		{\DTMitalianweekdayname{##4}}%
		\space
		\fi
		\fi
		\DTMifbool{italian}{showdayofmonth}%
		{%
			\DTMitalianordinal{##3}%
			\DTMitaliandaymonthsep
		}%
		{}%
		\DTMifbool{italian}{abbr}%
		{\DTMitalianshortmonthname{##2}}%
		{\DTMitalianmonthname{##2}}%
		\DTMifbool{italian}{showyear}%
		{%
			\DTMitalianmonthyearsep\number##1 % space intended
		}%
		{}%
	}%
	\renewcommand*\DTMDisplaydate[4]{%
		\ifDTMshowdow
		\ifnum##4>-1%
		\DTMifbool{italian}{abbr}%
		{\DTMitalianshortweekdayname{##4}}%
		{\DTMitalianweekdayname{##4}}%
		\space
		\fi
		\fi
		\DTMifbool{italian}{showdayofmonth}%
		{%
			\DTMitalianordinal{##3}%
			\DTMitaliandaymonthsep
		}%
		{}%
		\DTMifbool{italian}{abbr}%
		{\DTMitalianshortmonthname{##2}}%
		{\DTMitalianmonthname{##2}}%
		\DTMifbool{italian}{showyear}%
		{%
			\DTMitalianmonthyearsep\number##1 % space intended
		}%
		{}%
	}%
}%
{% time style
	\renewcommand*\DTMitaliantimesep{\DTMitaliantimesep}%
	\DTMsettimestyle{default}%
}%
{% zone style
	\DTMresetzones
	\DTMitalianzonemaps
	\renewcommand*{\DTMdisplayzone}[2]{%
		\DTMifbool{italian}{mapzone}%
		{\DTMusezonemapordefault{##1}{##2}}%
		{%
			\ifnum##1<0\else+\fi\DTMtwodigits{##1}%
			\ifDTMshowzoneminutes\DTMitaliantimesep\DTMtwodigits{##2}\fi
		}%
	}%
}%
{% full style
	\renewcommand*{\DTMdisplay}[9]{%
		\ifDTMshowdate
		\DTMdisplaydate{##1}{##2}{##3}{##4}%
		\DTMitaliandatetimesep
		\fi
		\DTMdisplaytime{##5}{##6}{##7}%
		\ifDTMshowzone
		\DTMitaliantimezonesep
		\DTMdisplayzone{##8}{##9}%
		\fi
	}%
	\renewcommand*{\DTMDisplay}[9]{%
		\ifDTMshowdate
		\DTMDisplaydate{##1}{##2}{##3}{##4}%
		\DTMitaliandatetimesep
		\fi
		\DTMdisplaytime{##5}{##6}{##7}%
		\ifDTMshowzone
		\DTMitaliantimezonesep
		\DTMdisplayzone{##8}{##9}%
		\fi
	}%
}%
%    \end{macrocode}
%
% Define numeric style.
%    \begin{macrocode}
\DTMnewstyle
 {italian-numeric}% label
 {% date style
    \renewcommand*\DTMdisplaydate[4]{%
      \DTMifbool{italian}{showdayofmonth}%
      {%
        \number##3 % space intended
        \DTMitaliandatesep
      }%
      {}%
      \number##2 % space intended
      \DTMifbool{italian}{showyear}%
      {%
        \DTMitaliandatesep
        \number##1 % space intended
      }%
      {}%
    }%
    \renewcommand*{\DTMDisplaydate}[4]{\DTMdisplaydate{##1}{##2}{##3}{##4}}%
 }%
 {% time style
    \renewcommand*\DTMdisplaytime[3]{%
      \number##1
      \DTMitaliantimesep\DTMtwodigits{##2}%
      \ifDTMshowseconds\DTMitaliantimesep\DTMtwodigits{##3}\fi
    }%
 }%
 {% zone style
   \DTMresetzones
   \DTMitalianzonemaps
   \renewcommand*{\DTMdisplayzone}[2]{%
     \DTMifbool{italian}{mapzone}%
     {\DTMusezonemapordefault{##1}{##2}}%
     {%
       \ifnum##1<0\else+\fi\DTMtwodigits{##1}%
       \ifDTMshowzoneminutes\DTMitaliantimesep\DTMtwodigits{##2}\fi
     }%
   }%
 }%
 {% full style
   \renewcommand*{\DTMdisplay}[9]{%
    \ifDTMshowdate
     \DTMdisplaydate{##1}{##2}{##3}{##4}%
     \DTMitaliandatetimesep
    \fi
    \DTMdisplaytime{##5}{##6}{##7}%
    \ifDTMshowzone
     \DTMitaliantimezonesep
     \DTMdisplayzone{##8}{##9}%
    \fi
   }%
   \renewcommand*{\DTMDisplay}{\DTMdisplay}%
 }
%    \end{macrocode}
%
% Define am pm style.
%    \begin{macrocode}
 \DTMnewtimestyle
 {italianampm}% label
 {%
 	\renewcommand*\DTMdisplaytime[3]{%
 		\ifnum##2=0
 		\ifnum##1=12
 		\DTMtexorpdfstring
 		{\DTMitalianampmfmt{\DTMitaliannoon}}%
 		{\DTMitaliannoon}%
 		\else
 		\ifnum##1=0
 		\DTMtexorpdfstring
 		{\DTMitalianampmfmt{\DTMitalianmidnight}}%
 		{\DTMitalianmidnight}%
 		\else
 		\ifnum##1=24
 		\DTMtexorpdfstring
 		{\DTMitalianampmfmt{\DTMitalianmidnight}}%
 		{\DTMitalianmidnight}%
 		\else
 		\ifnum##1<12
 		\number##1
 		\DTMtexorpdfstring
 		{\DTMitalianampmfmt{\DTMitalianam}}%
 		{\DTMitalianam}%
 		\else
 		\number\numexpr##1-12\relax
 		\DTMtexorpdfstring
 		{\DTMitalianampmfmt{\DTMitalianpm}}%
 		{\DTMitalianpm}%
 		\fi
 		\fi
 		\fi
 		\fi
 		\else
 		\ifnum##1<13
 		\ifnum##1=0
 		12%
 		\else
 		\number##1
 		\fi
 		\DTMitaliantimesep\DTMtwodigits{##2}%
 		\ifnum##1=12
 		\DTMtexorpdfstring
 		{\DTMitalianampmfmt{\DTMitalianam}}%
 		{\DTMitalianpm}%
 		\else
 		\DTMtexorpdfstring
 		{\DTMitalianampmfmt{\DTMitalianam}}%
 		{\DTMitalianam}%
 		\fi
 		\else
 		\number\numexpr##1-12\relax
 		\DTMitaliantimesep\DTMtwodigits{##2}%
 		\ifnum##1=24
 		\DTMtexorpdfstring
 		{\DTMitalianampmfmt{\DTMitalianpm}}%
 		{\DTMitalianam}%
 		\else
 		\DTMtexorpdfstring
 		{\DTMitalianampmfmt{\DTMitalianpm}}%
 		{\DTMitalianpm}%
 		\fi
 		\fi
 		\fi
 	}%
 }%
%    \end{macrocode}
%
% Define italian-twenty-four style.
%
%    \begin{macrocode}
 \DTMnewtimestyle
 {italian-twenty-four}% label
 {%
 	\renewcommand*\DTMdisplaytime[3]{%
 		\ifnum##2=0
 		\ifnum##1=12
 		\DTMtexorpdfstring
 		{\DTMitalianampmfmt{\DTMitaliannoon}}%
 		{\DTMitaliannoon}%
 		\else
 		\ifnum##1=0
 		\DTMtexorpdfstring
 		{\DTMitalianampmfmt{\DTMitalianmidnight}}%
 		{\DTMitalianmidnight}%
 		\else
 		\ifnum##1=24
 		\DTMtexorpdfstring
 		{\DTMitalianampmfmt{\DTMitalianmidnight}}%
 		{\DTMitalianmidnight}%
 		\else
 		\ifnum##1<12
 		\DTMtwodigits{##1}\DTMitaliantimesep\DTMtwodigits{##2}%
 		\else
 		\DTMtwodigits{##1}\DTMitaliantimesep\DTMtwodigits{##2}%
 		\fi
 		\fi
 		\fi
 		\fi
 		\else
 		\ifnum##1<13
 		\ifnum##1=0 
 		\DTMtwodigits{##1}%
 		\else
 		\DTMtwodigits{##1}\relax
 		\fi
 		\DTMitaliantimesep\DTMtwodigits{##2}%
 		\else
 		\DTMtwodigits{##1}\relax
 		\DTMitaliantimesep\DTMtwodigits{##2}%
 		\fi
 		\fi
 	}%
 }%
%    \end{macrocode}
%
%
%\begin{macro}{\DTMitalianzonemaps}
% The time zone mappings are set through this command, which can be
% redefined if extra mappings are required or mappings need to be
% removed.
%    \begin{macrocode}
\newcommand*{\DTMitalianzonemaps}{%
  \DTMdefzonemap{01}{00}{CET}%
  \DTMdefzonemap{02}{00}{CEST}%
}
%    \end{macrocode}
%\end{macro}
% Switch style according to the \opt{useregional} setting.
%    \begin{macrocode}
\DTMifcaseregional
{}% do nothing
{\DTMsetstyle{italian}}
{\DTMsetstyle{italian-numeric}}
%    \end{macrocode}
%
% Redefine \cs{dateitalian} (or \cs{date}\meta{dialect}) to prevent
% \sty{babel} from resetting \cs{today}. (For this to work,
% \sty{babel} must already have been loaded if it's required.)
%    \begin{macrocode}
\ifcsundef{date\CurrentTrackedDialect}
{%
  \ifundef\dateitalian
  {% do nothing
  }%
  {%
    \def\dateitalian{%
      \DTMifcaseregional
      {}% do nothing
      {\DTMsetstyle{italian}}%
      {\DTMsetstyle{italian-numeric}}%
    }%
  }%
}%
{%
  \csdef{date\CurrentTrackedDialect}{%
    \DTMifcaseregional
    {}% do nothing
    {\DTMsetstyle{italian}}%
    {\DTMsetstyle{italian-numeric}}
  }%
}%
%    \end{macrocode}
%\iffalse
%    \begin{macrocode}
%</datetime2-italian.ldf>
%    \end{macrocode}
%\fi
%\Finale
\endinput
